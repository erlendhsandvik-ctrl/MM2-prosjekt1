\section{Teori}
\subsection{Bølgeligningen}

    Bølgeligningen er en partielldifferensialligning som beskriver hvordan bølger, som lyd, lys eller vannbølger, forplanter seg gjennom et medium. Den én-dimensjonale bølgeligningen har formen
    \[
    \frac{\partial^2 u}{\partial t^2} = c^2 \frac{\partial^2 u}{\partial x^2}
    \]
    hvor $u(x,t)$ beskriver bølgens utslag ved posisjon $x$ og tid $t$, og $c$ er bølgefarten.

    Betydningen av bølgeligningen er at den gir et matematisk rammeverk for å forstå og forutsi hvordan bølger beveger seg og endrer seg over tid. Løsninger til bølgeligningen brukes i mange fagfelt, inkludert fysikk, ingeniørfag og akustikk.
 
    \subsubsection{Utledning av bølgeligningen}

        FLTTE TIL METODE?  Avsnittet er 11.2 i bølgeligning pdf fil

        Vi betrakter en elastisk streng av lengde $L$, som er spent fast i begge ender. Strengen
        har konstant massetetthet $\rho$ og utsettes for en konstant spenning $T$. 
        Vi antar små utslag i et vertikalt plan, slik at helninger og forskyvninger
        kan tilnærmes som små. 

        La $u(x,t)$ beskrive den vertikale forskyvningen i punktet $x$ til tid $t$.
        Vi ser på et lite stykke av strengen mellom $x$ og $x+\Delta x$. I endepunktene virker
        tensjonskreftene $T_1$ og $T_2$ langs tangentene til strengen. De horisontale
        komponentene kansellerer, slik at vi kun trenger å se på de vertikale komponentene.
        Summen av de vertikale kreftene er
        \[
        F \approx T \sin\beta - T \sin\alpha,
        \]
        der $\alpha$ og $\beta$ er vinklene til strengen ved $x$ og $x+\Delta x$.

        For små vinkler gjelder $\tan\theta \approx \sin\theta \approx \tfrac{\partial u}{\partial x}$.
        Dermed kan kraften skrives som
        \[
        F \approx T\left( \frac{\partial u}{\partial x}(x+\Delta x,t) -
                        \frac{\partial u}{\partial x}(x,t)\right).
        \]

        Utvikler vi dette videre får vi
        \[
        F \approx T \frac{\partial^2 u}{\partial x^2}(x,t)\,\Delta x.
        \]

        Ifølge Newtons 2.\ lov må denne kraften være lik massen av elementet
        $\rho \Delta x$ multiplisert med akselerasjonen:
        \[
        \rho \Delta x \, \frac{\partial^2 u}{\partial t^2}(x,t)
        = T \frac{\partial^2 u}{\partial x^2}(x,t)\,\Delta x.
        \]

        Etter forkorting av $\Delta x$ får vi den klassiske bølgeligningen
        \[
        \frac{\partial^2 u}{\partial t^2} = c^2 \frac{\partial^2 u}{\partial x^2},
        \]
        der
        \[
        c = \sqrt{\frac{T}{\rho}}
        \]
        er bølgefarten, bestemt av forholdet mellom spenningen $T$ og massetettheten $\rho$.

        Denne ligningen beskriver små vibrasjoner i en streng med faste endepunkter.

\subsection{Fourierrekker}
    Fourierrekker er en metode for å representere periodiske funksjoner som en sum av sinus- og cosinusfunksjoner. Dette er spesielt nyttig i løsningen av partielldifferensialligninger som bølgeligningen, hvor komplekse bølgeformer kan brytes ned i enklere harmoniske komponenter.    
    
    Grunnleggende konsepter inkluderer:

    Periodiske funksjoner

    Ortogonalitet og basisfunksjoner

    Koeffisientene i en Fourierrekke
\subsection{Løsningsmetoder}

\subsubsection{Separasjonsmetoden}

    Separasjonsmetoden er en teknikk for å løse partielle differensialligninger (PDE-er) som er lineære og homogene. Ideen er å anta at løsningen kan skrives som et produkt av funksjoner der hver funksjon bare avhenger av én variabel:
    \[
    u(x,t) = X(x)\,T(t).
    \]

    Ved å sette dette inn i PDE-en kan man ofte dele opp ligningen slik at den romlige delen og den tidsavhengige delen blir atskilt. Dette gir en konstant, kalt \emph{separasjonskonstanten}:
    \[
    \frac{T'(t)}{T(t)} = \frac{X''(x)}{X(x)} = -\lambda.
    \]

    Dermed reduseres PDE-en til to ordinære differensialligninger , en for $X(x)$ og en for $T(t)$. Randbetingelser brukes til å bestemme mulige verdier av $\lambda$, som gir en mengde \emph{egenfunksjoner} $X(x)$. Den generelle løsningen blir en superposisjon av slike separable løsninger:
    \[
    u(x,t) = \sum_{n=1}^\infty c_n X_n(x) T_n(t).
    \]

    Denne metoden er spesielt nyttig i kombinasjon med Fourier-rekker, siden egenfunksjonene ofte er sinus- og cosinusfunksjoner som danner en ortogonal basis.


    Fouriermetoden

    Eventuelt D’Alemberts løsning
